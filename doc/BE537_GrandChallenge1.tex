\documentclass{article}
\usepackage{graphicx}
\usepackage{color}
\usepackage{listings}
\usepackage{mcode}
\usepackage{fullpage}
\usepackage{hyperref}
\usepackage{amsmath}
\usepackage{blindtext}

\definecolor{lightgray}{gray}{0.5}
\setlength{\parindent}{0pt}

\begin{document}    

	\begin{par}
	
		\title{BE 537 - Grand Challenge 1}		
		\date{\today}
		\author{James Wang, Michael Lautman, Shreel Vijayvargiya}
		\maketitle
	
	\end{par}

	\begin{par}
		\section*{Introduction}
		For this project we explored methods for performing groupwise registration between a set of brain images. We performed registration on a set of 3D brain images solving for transformations from the set of images into a common reference space. The quality of a registration was established by measuring how closely a set of labeled features in the images corresponded when projected into the common frame via the transformations we built.
	\end{par}
	
	\begin{par}
		\section*{3.1 The Basic Component}
		To configure the environment in matlab we have included a setup script.
		\lstinputlisting{../setup_env.m}
		
		\subsection*{3.1.1 Extend myView to Display Registration Results}
			Our project uses a visualizer that takes in a fixed image, a moving image, the voxel spacing in the image, a rotation matrix and a translation vector. 
			\begin{lstlisting}
function myViewAffineReg(fixed, moving, spacing, A, b)
			\end{lstlisting}
			\lstinputlisting{../myViewAffineReg.m}
			
		\subsection*{3.1.2 3D Affine Registration Objective Function}
			We compute the objective function for 3D registration using the equation below.
			\begin{align*}
				E(A,b) = \int_{\Omega} [ I(x) - J(Ax+b)]^{2} dx
			\end{align*}

			\begin{lstlisting}
function [E,g] = myAffineObjective3D(p,I,J,varargin)
			\end{lstlisting}
			
			\lstinputlisting{../myAffineObjective3D.m}

		\subsection*{3.1.3 Testing the Correctness of Gradient Computation}
			We verify our analytic gradient computation by computing a numerical gradient approximation that utilizes the central finite difference approximation. 
			\begin{align*}
				\frac{\partial E}{\partial p_{j}} |_{p} \simeq  \frac{E(p + \epsilon e_{j}) - E(p - \epsilon e_{j})}{2 \epsilon}
			\end{align*}
			We then test our numerical gradient computation on an image \lstinline|'sub001_mri.nii'|. Our script shows that The maximum relative error found using an $\epsilon = 1e-4$ and $\sigma = 1$ is  $\approx 0.52\%$.
			
			
			\lstinputlisting{../p313.m}
		
		\subsection*{3.1.4 Testing our objective function by registering two images}
		
		\lstinputlisting{../p314.m}
		
		Running our solution on an example 
	\end{par}
	

\end{document}
    
